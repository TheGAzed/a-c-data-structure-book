%%%%%%%%%%%%%%%%%%%% book.tex %%%%%%%%%%%%%%%%%%%%%%%%%%%%%
%
% sample root file for the chapters of your monograph or textbook
%
% Use this file as a template for your own input.
%
%%%%%%%%%%%%%%%% Springer Nature %%%%%%%%%%%%%%%%%%%%%%%%%%


% RECOMMENDED %%%%%%%%%%%%%%%%%%%%%%%%%%%%%%%%%%%%%%%%%%%%%%%%%%%
\documentclass[graybox,envcountchap,sectrefs]{SNmono}

%\usepackage{mathptmx}
%\usepackage{helvet}
%\usepackage{courier}
\usepackage{type1cm}         
\usepackage{hyperref}

\usepackage{fontspec}
\usepackage{xcolor}
\usepackage{listings}
\usepackage[figuresleft]{rotating}

\definecolor{comment}{RGB}{2, 90, 10}
\definecolor{keyword}{RGB}{140, 58, 140}
\definecolor{string}{RGB}{98, 41, 2}
\definecolor{directive}{RGB}{45, 75, 150}

\newfontfamily\codefont{JetBrains Mono}[Scale=0.9]

\lstdefinestyle{VisualStudio}{
    language=C,
    basicstyle=\codefont\small,
    literate={*}{{{\char42}}}1,
    columns=flexible,
    keywordstyle=\color{keyword},
    commentstyle=\color{comment},
    stringstyle=\color{string},
    emph={int,char,double,float,unsigned,void,bool, 
    istack_s, iqueue_s, ideque_s,},
    emphstyle={\color{directive}},
    directivestyle=\color{directive},
    breaklines=true,
    keepspaces=true,
    showstringspaces=false,
    frame=lines,
    framesep=5pt,
    framerule=0.4pt,
}

\usepackage[most]{tcolorbox}

\newtcolorbox{infobox}[1][]{
  colback=black!10!white,
  colframe=white!10!black,
  fonttitle=\bfseries,
  arc=0mm,
  breakable, % Allows box to break across pages
  #1
}

\usepackage{makeidx}         % allows index generation
\usepackage{graphicx}        % standard LaTeX graphics tool
                             % when including figure files
\usepackage{multicol}        % used for the two-column index
\usepackage[bottom]{footmisc}% places footnotes at page bottom

\usepackage{newtxtext}       % 
\usepackage[varvw]{newtxmath}       % selects Times Roman as basic font

\makeindex             % used for the subject index
                       % please use the style svind.ist with
                       % your makeindex program

%%%%%%%%%%%%%%%%%%%%%%%%%%%%%%%%%%%%%%%%%%%%%%%%%%%%%%%%%%%%%%%%%%%%%

\begin{document}

\author{Matej Dedina}
\title{A C Data Structure Book}
\subtitle{BOOK I}
\maketitle

\frontmatter%%%%%%%%%%%%%%%%%%%%%%%%%%%%%%%%%%%%%%%%%%%%%%%%%%%%%%

% \include{dedication}
% \include{foreword}

\chapter*{Preface}

This book describes the process of designing and implementing a personal data structure library in the C programming language. Each implementation is centered around two types of implementations - an infinitely expandable structure able to hold any amount of elements, and a finite one able to hold only a certain maximum amount.

The goal of this work is not to be the go-to data structure book for C, on the contrary, the name is meant to represent an affordable book for the public which doesn't take itself seriously.

The book will be divided into parts classifying all structures into categories of chapters. Each chapter will explain the process of creating said structure and the code used to make it. 

Since the code is in C a decent understanding of the programming language is required, that primarily includes - structures (struct), pointers, memory allocation, and last but certainly not least - function pointers.

All the source code will be available at github under the {\texttt{Unlicense License}}.

If I have to put a disclaimer I just want to clarify that the reason why some paragraphs may be written like this is because the book was created using \LaTeX in Visual Studio Code and I hate seeing {\texttt{Overfull \textbackslash hbox}} highlights.

% \include{acknow}
% \include{ethics}

\tableofcontents

% \include{acronym}

\mainmatter%%%%%%%%%%%%%%%%%%%%%%%%%%%%%%%%%%%%%%%%%%%%%%%%%%%%%%%
\chapter{The cerpec data structure library}

The name \texttt{cerpec} may look familiar if you're a Polish speaker, but the word actually comes from the Eastern Slovak word for suffering. One can also see the similarities when writing the standard Slovak, Eastern Slovak, and Polish words next to each other, like this \texttt{trpieť - cerpec - cierpieć}.

The reasons for choosing this name are three-fold; making this library was like suffering through psychological pain; Eastern Slovak infinitives ending with -c, standard Slovak with -ť, go nicely with the C programming language; and I doubt anyone would use this word for trademarking reasons since it's a dialectic word.

\section{The main header}

The \texttt{cerpec.h} serves as the parent containing all the necessary definitions shared among all the data structures. The header can be divided into four parts.

\subsection{Growth factor and chunk size}

The growth factor \texttt{CERPEC\_FACTOR} represents the factor by which to determine the previous and next memory size to resize into when we're either out of space or if smaller space is available. 

\begin{lstlisting}[language=C, style=VisualStudio, label=lst:create-stack]
#define CERPEC_FACTOR 2
\end{lstlisting}

The chunk size \texttt{CERPEC\_CHUNK} is used to avoid a small starting size where the program might be too busy just reallocating the structures instead of performing operations \ref{fig:growth}. Basically, instead of starting at length 1 after inserting the first element, it starts at a specific power of two, in this case 256, and grows exponentially.

\begin{lstlisting}[language=C, style=VisualStudio, label=lst:factor-chunk]
// define chunk size to expand and contract all data structures
#if !defined(CERPEC_CHUNK)
#   define CERPEC_CHUNK 256
#elif CERPEC_CHUNK <= 0
#   error "Chunk size must be greater than zero."
#elif (CERPEC_CHUNK & (CERPEC_CHUNK - 1))
#   error "Chunk size must be a power of 2."
#endif
\end{lstlisting}

\begin{figure}
    \centering
    \includegraphics[width=1.0\textwidth]{figures/growth}
    \caption{Example growth if chunk size is 1 (top) vs 8 (bottom). The height of the pyramid represents the number of memory growths.}
    \label{fig:growth}
\end{figure}

\subsection{Error and validation handling}

Assertions are a simple way to check if a property is true either before or after an action has been performed. For \texttt{cerpec} there're two types of assertions - \texttt{error} and \texttt{valid}.

The \texttt{error} assertion errors when something bad that shouldn't happen happens, for example - a parameter being NULL, memory allocation failing, no element being found. The \texttt{valid} assertion validates the state of the structure, for example if \texttt{length} member is less than or equal to \texttt{capacity} member, if element \texttt{size} member is greater than zero. 

It's just a means to allow the user to better specify which assertions they want to check and disable. The way to disable \texttt{valid}, \texttt{error} and all assertions altogether are via defining \texttt{NVALID}, \texttt{NERROR} and \texttt{NDEBUG} either through compiler flag macro definitions or using \texttt{\#define} prior to any data structure \texttt{\#include}. I believe that in the future more assertions will be added.

\break

\begin{lstlisting}[language=C, style=VisualStudio, label=lst:assert]
#ifdef NDEBUG
#   define NVALID
#   define NERROR
#endif

#ifndef NVALID
#   define valid(condition) assert(condition)
#else
#   define valid(condition) (void)(0)
#endif

#ifndef NERROR
#   define error(condition) assert(condition)
#else
#   define error(condition) (void)(0)
#endif
\end{lstlisting}

\subsection{Custom memory allocator}

Each structure ahs a pointer to a custom memory allocator which can be used to allocate, free and reallocate memory based on arguments. The \texttt{standard} memory allocator uses the standard library's \texttt{malloc}, \texttt{free} and \texttt{realloc} functions.

\begin{infobox}[title=Linux manual page - malloc(3)]
The \texttt{malloc()} function allocates size bytes and returns a pointer to the allocated memory.  The memory is not initialized.  If size is 0, then \texttt{malloc()} returns a unique pointer value that can later be successfully passed to \texttt{free()}.
\end{infobox}

\begin{infobox}[title=Linux manual page - free(3)]
The \texttt{free()} function frees the memory space pointed to by p, which must have been returned by a previous call to \texttt{malloc()} or related functions.  Otherwise, or if p has already been freed, undefined behavior occurs.  If p is NULL, no operation is performed.
\end{infobox}

\begin{infobox}[title=Linux manual page - realloc(3)]
The \texttt{realloc()} function changes the size of the memory block pointed to by p to size bytes. The contents of the memory will be unchanged in the range from the start of the region up to the minimum of the old and new sizes.  If the new size is larger than the old size, the added memory will not be initialized.

If p is NULL, then the call is equivalent to \texttt{malloc(size)}, for all values of size.

If size is equal to zero, and p is not NULL, then the call is equivalent to \texttt{free(p)}.

Unless p is NULL, it must have been returned by an earlier call to malloc or related functions.  If the area pointed to was moved, a free(p) is done.
\end{infobox}

The only way to use custom memory is through defining three custom function pointers and putting them as parameters for the \texttt{compose\_memory} builder:

\begin{lstlisting}[language=C, style=VisualStudio, label=lst:allocator]
typedef void * (*alloc_fn)   (size_t const, void *);
typedef void * (*realloc_fn) (void *, size_t const, void *);
typedef void   (*free_fn)    (void *, void *);

typedef struct memory {
    void * arguments;
    alloc_fn alloc;
    realloc_fn realloc;
    free_fn free;
} memory_s;

memory_s compose_memory(alloc_fn const alloc, realloc_fn const realloc, free_fn const free, void * const arguments);
\end{lstlisting}

The standard memory allocations can be accessed through the constant external variable \texttt{standard}.

\begin{lstlisting}[language=C, style=VisualStudio, label=lst:standard]
extern const memory_s standard;
\end{lstlisting}

\subsection{Function pointers}

This is a list of all available function pointers which allow for more generic data manipulation.

\begin{lstlisting}[language=C, style=VisualStudio, label=lst:allocator]
typedef void   (*set_fn)     (void * const element);
typedef void * (*copy_fn)    (void * const destination, void const * const source);
typedef size_t (*hash_fn)    (void const * const element);
typedef int    (*compare_fn) (void const * const a, void const * const b);
typedef bool   (*filter_fn)  (void const * const element);
typedef bool   (*handle_fn)  (void * const element, void * const arguments);
typedef void   (*process_fn) (void * const array, size_t const lenght, void * const arguments);
typedef void   (*operate_fn) (void * const result, void const * const a, void const * const b);
\end{lstlisting}

\begin{enumerate}
    \item \texttt{set\_fn} - a function pointer that allows to manipulate a generic element pointer directly, can be used to set the specified element to zero, deallocate memory, increment and decrement value.
    \item \texttt{copy\_fn} - a function pointer which allows to create with all nested sub-elements recreated deeply or referenced shallowly from a source element reference into a destination one.
    \item \texttt{hash\_fn} - a function pointer which takes in a generic element and returns a \texttt{size\_t} value representing a hash. This hash can be used to index the value into an array.
    \item \texttt{compare\_fn} - a function pointer that compares elements \texttt{a} and \texttt{b}, and returns 0 if equal, negative number if \texttt{a} is smaller, and positive otherwise.
    \item \texttt{filter\_fn} - a function pointer that returns \texttt{true} if the specified element meets a certain criteria, \texttt{false} otherwise. Can be used to extract element in linear list structures, for example odd/even numbers or primes.
    \item \texttt{handle\_fn} - a function pointer that can handle a single element based on arguments. This function allows to iteratively chang elements until \texttt{false} is returned (kinda like a \texttt{break} in a classic c loop.).
    \item \texttt{process\_fn} - a function pointer that allows to process an array of elements instead of iteratively walking through each one. Can be used to sort a sequence of element in array form for linear data structures like stacks, queues and lists.
    \item \texttt{operate\_fn} - a function pointer that performs an operation on two elements and saves the result inside the first \texttt{result} parameter. The only usage is in the \(F(n) = G(n) + H(n)\) calculation for the A* path finding algorithm (\(F(n)\) being \texttt{result} and the rest being the other two parameters).
\end{enumerate}


\begin{partbacktext}

\part{Restricted sequential data structures} \label{p:one}
\noindent The name of the first part comes from the inability to find a categoric name for only stacks, queues and deques. 
Online sources categorize those as linear dynamic data structures, the problem is... linked lists are also part of this category. 
I want to put list implementations into a separate book part, thus I hereby coin the term {\normalfont\texttt{Restricted sequential data structures}} as a subcategory ob dynamic linear data structures.
I know nobody will take this declaration seriously, but for the sake of this book please bear with it.

\end{partbacktext}

\chapter{Stacks}

Lets imagine we have a stack of books. The normal way we can add another one to the stack is by putting it at the top, if we put the book with the cover pointing upwards one is able to know what book exactly is at the top. To remove a book we just grab the top one and put it somewhere else.

By making space for another stack of books and adding all the books from the first one while the constrains of putting and removing still remain we get a new stack, but with a specific property that, when compared with our initial one, makes us tilt our head (literally), because the new one is just like the last one, but with the books in opposite order. This little property allows us to invert the order of any linear data structure without knowing how it's implemented, if the structure allows us to add and remove elements.

Continuing with the two stack analogy, what if instead of books we stored a game of chess. By adding these moves one-by-one \ref{fig:chess-moves}: E2-E4, E7-E5, G1-F3, B8-C6, D2-D4, E5-D4, F3-D4, F8-C5, C2-C3, D8-F6, D4-C6, F6-F2; we get probably the funniest chess game in modern history \footnote{that can be watched via \url{https://www.youtube.com/watch?v=e91M0XLX7Jw}}. And on top of that if we remove the two moves from the top and reset the pieces (F6-F2 becomes F2-F6), iteratively it is possible to return to the initial state of the game. And to top it all of, as probably every data structure book and article about stacks writes this mechanism is used by search engines when you want to return to the previous pages and back.

\begin{figure}[h!]
    \centering
    \includegraphics[width=1.0\textwidth]{figures/stack/chess-moves}
    \caption{The entire game of chess between content creators xQc and MoistCr1tikal.}
    \label{fig:chess-moves}
\end{figure}

Stacks are also used to store function calls in programming languages like C to be able to know to what previous in memory place the program should go to after the top function returns.

\section{Implementations of a Stack}

Every computer science student learns that the two most common ways to implement a stack in C are:

\begin{enumerate}
    \item Singly linked lists
    \item Static/dynamic arrays
\end{enumerate}

So why does this book have three subsections for stack implementations?

\subsection{The humble linked list}

For a more in-depth explanation of lists you can infer to PART \ref{p:two} \nameref{p:two}, but in short a linked list is a linear data structure which allows us to store data sequentially as well as store a reference to the next data item \ref{fig:linked-list}.

The linked list consists of two special nodes which represent the beginning and the end - head and tail. Most linked list implementation will store the reference to the head, so it is generally easy to access the first element as oppose to the tail, since the user needs to follow all the arrows until they reach it.

\begin{figure}[h!]
    \centering
    \includegraphics[width=1.0\textwidth]{figures/stack/linked-list}
    \caption{A simple graphical linked list example.}
    \label{fig:linked-list}
\end{figure}

\section{Dynamic Array Stack Structure}

\chapter{Queue} \label{ch:queue}

Imagine you're waiting for your daily calory intake, lunch, in the school and you see a queue. When one person gets their lunch the entire queue moves one person to the front, because, logically, the entire cafeteria won't move to the next person.

But that's NOT how queues in programming work. If each person is a single element and the cafeteria is at the current place of the first person in the queue, then you don't want to move each element to the cafeteria, but move the cafeteria itself. So we have a current pointer/index/position which, after getting rid of the first element, we go/increment/move to the second one, and so on.

Going back to \autoref{ch:stack} \nameref{ch:stack}'s chess analogy, when we put those moves into the stack we're kind of pulling them from an array of moves from the front tto the back. Now lets say there're move moves going on, plus we want to play back and forth the moves which were already played without waiting for the game to finish. To solve this we can pipe the moves into a special queue, then push elements from the queue into stack one. If we want to go to the beginning, then we push the moves from the stack one into stack two. To play all the way to the latest move, all that's left is to first push all the elements from stack two into stack one, and push all moves from the queue into stack one. The top element in stack is the latest move, see \autoref{fig:chess-queue-stack}.

\begin{figure}[p]
    \centering
    \includegraphics[angle=90, height=0.965\textheight]{figures/queue/chess-queue-stack}
    \caption{Visual example of two stacks and a queue being used to track the chess game.}
    \label{fig:chess-queue-stack}
\end{figure}

\section{Implementations of a queue}

Similarly to stacks, there are also two ways to implement queues:

\begin{itemize}
    \item Singly linked lists
    \item Static arrays (not dynamic)
\end{itemize}

\subsection{The free linked list}

Usually when implementing a queue using linked lists we create a list structure that ahs one reference to the head, for access and removal; and to the tail node, for pushing elements. 

However, there is also a slightly better way to implement them, which is to use circular lists. The next subsection will talk about circular arrays, but we can also use circular lists, kinda like \autoref{fig:circular-list}. Hear me out, this won't work if we only have a reference to the head, so lets go with the next best thing - the tail. The list will be circular, thus the tail's next node will be the head. 

\begin{figure}
    \centering
    \includegraphics[width=0.6\textwidth]{figures/queue/circular-list}
    \caption{Visual representation of a circular singly linked list.}
    \label{fig:circular-list}
\end{figure}

This makes access and removal as simple as getting the the head from the tail, also adding elements is easy since we just make the new node's next reference the head and change tail's next to new node, see \autoref{fig:list-queue}. The downside is that circular linked lists are more complex than standard, but by just having one, the tail, node pointer we can decrease the size of the structure a bit (or a whopping 32/64 bits at that).

\begin{figure}
    \centering
    \includegraphics[width=\textwidth]{figures/queue/list-queue}
    \caption{Example of adding (top) and removing (bottom) an element from a circular singly linked list queue.}
    \label{fig:list-queue}
\end{figure}

\chapter{Deque} \label{ch:deque}


\begin{partbacktext}

\part{Linked lists} \label{p:two}
\noindent 
\end{partbacktext}


%\include{appendix}

%\backmatter%%%%%%%%%%%%%%%%%%%%%%%%%%%%%%%%%%%%%%%%%%%%%%%%%%%%%%%
%\include{glossary}
%\include{solutions}
\printindex

%%%%%%%%%%%%%%%%%%%%%%%%%%%%%%%%%%%%%%%%%%%%%%%%%%%%%%%%%%%%%%%%%%%%%%

\end{document}




